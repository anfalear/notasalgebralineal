\chapter{Políticas del curso}

\noindent\fbox{\parbox{\textwidth}{\textit{
La información aquí presentada no se modificará por ninguna de las partes a menos que haya un consenso entre estas, el cual debe ser oportunamente informado y reportado por escrito en este documento. Ante cualquier instancia, este documento será soporte para las acciones a las que se de lugar, por lo cual, tanto el profesor como el estudiante lo aceptan y se someten a lo que esté aquí escrito. 
}}}

Esta política se rige de acuerdo a lo dispuesto en los Reglamentos Académicos de Pregrado de cada institución (R.A.P.) disponibles en las web institucionales.

Los cursos de Álgebra Lineal tendrán su Aula Virtual, donde se informarán horarios de atención, distribución de exámenes, demás aspectos particulares, calificaciones y distribuciones de porcentajes. La nota mínima de aprobación es 3.0.

\section{Contenido Álgebra Lineal} 

La cobertura total del contenido dependerá del progreso del semestre; es responsabilidad del estudiante mantenerse al día. Los temas evaluados en cada corte se proporcionarán oportunamente. Los temas marcados con asterisco (*) son opcionales según el progreso del semestre.

\begin{enumerate}
\item \textbf{El campo de los números complejos}
\begin{enumerate}
\item Orígenes, definición y representaciones (geométrica, binomial y polar).
\item Aritmética: suma, producto, potencias y raíces. Propiedades.
\item Ecuaciones polinómicas y Teorema Fundamental del álgebra.
\end{enumerate}

\item \textbf{Matrices}
\begin{enumerate}
\item Álgebra de matrices: suma, producto por escalar y producto de matrices.
\item Operaciones elementales entre filas. Matrices elementales, equivalentes por filas y escalonadas reducidas.
\item Matrices invertibles. Algoritmo para encontrar la inversa.
\end{enumerate}

\item \textbf{Determinantes}
\begin{enumerate}
\item Definición. Fórmulas de expansión (método de cofactores).
\item Cálculo por diagonalización.
\item Propiedades: fórmula del producto, determinante de la transpuesta e inversa. Fórmula de la adjunta.
\end{enumerate}

\item \textbf{Sistemas de ecuaciones lineales}
\begin{enumerate}
\item Definición, ejemplos y representación matricial.
\item Solución general (método de eliminación de Gauss).
\item Regla de Cramer.
\item Condiciones de existencia y unicidad. Sistemas homogéneos.
\item Problemas de aplicación.
\end{enumerate}

\item \textbf{$\mathbb{R}^n$ como espacio vectorial euclidiano}
\begin{enumerate}
\item Vectores en $\mathbb{R}^2$ y $\mathbb{R}^3$. Representación geométrica y algebraica.
\item Álgebra de vectores: suma, producto por escalar. Propiedades de $\mathbb{R}^n$.
\item Producto escalar, vectorial, proyecciones, ángulos, normas, distancias, ortogonalidad.
\item Rectas en el espacio.
\item Planos en el espacio.
\end{enumerate}

\item \textbf{Espacios vectoriales}
\begin{enumerate}[$a)$]
\item Definición y ejemplos.
\item Subespacios vectoriales.
\item Combinaciones lineales: subespacio generado, sistemas generadores.
\item Dependencia e independencia lineal. Bases y dimensión.
\item Coordenadas y cambios de bases.
\item Espacios fundamentales: fila, columna y nulo; nulidad, rango y teorema de la dimensión.
\end{enumerate}

\item \textbf{Espacios con producto interno}
\begin{enumerate}[$a)$]
\item Definición y ejemplos.
\item Normas, ángulos, distancias, desigualdades triangular y de Cauchy-Schwarz, teorema de Pitágoras.
\item Conjuntos ortogonales y ortonormales. Proceso de Gram-Schmidt.
\item Proyecciones ortogonales. Complementos ortogonales. Teorema de descomposición ortogonal.
\item Aplicaciones: mínimos cuadrados.
\item Aplicaciones: aproximación por series de Fourier.
\end{enumerate}

\item \textbf{Valores y vectores propios}
\begin{enumerate}[$a)$]
\item Definición, ejemplos y polinomio característico.
\item Subespacios propios: multiplicidad geométrica y algebraica.
\item Diagonalización: matrices semejantes y simétricas.
\item Teorema de Cayley-Hamilton.
\item Aplicaciones: potenciación, formas cuadráticas, ecuaciones diferenciales, sistemas dinámicos y cadenas de Markov.
\end{enumerate}

\item \textbf{Transformaciones lineales}
\begin{enumerate}[$a)$]
\item Definición, ejemplos y propiedades básicas.
\item Transformaciones en el plano y espacio. Geometría en $\mathbb{R}^2$ y $\mathbb{R}^3$: rotación, estiramientos, deslizamientos.
\item Núcleo e imagen: rango, nulidad y teorema de la dimensión.
\item Representación matricial.
\item Álgebra de operadores. Composición y multiplicación de matrices.
\item Isomorfismos: transformación inversa.
\end{enumerate}
\end{enumerate}

\textbf{Notas aclaratorias:}
\begin{itemize}
\item No hay número específico de quices o trabajos; si son varios se promedian.
\item Las calificaciones se publican en la plataforma universitaria con notificación por correo. Si el estudiante no puede visualizar su nota, debe manifestarlo por correo al profesor.
\item El profesor acordará con los estudiantes la forma de revisar trabajos y exámenes.
\end{itemize}

\section{Políticas de clase}

\begin{itemize}
\item Los instrumentos electrónicos requieren aprobación del profesor.

\item Quices y exámenes sin apuntes, libros ni aparatos electrónicos (salvo flexibilización informada). El estudiante debe leer las reglas de cada evaluación.

\item El trabajo debe ser autónomo o en grupo designado. \textbf{Ofrecer} o \textbf{aceptar} soluciones ajenas constituye \textbf{plagio}, penalizado según el R.A.P. Incluye omisión de fuentes e uso no citado de IA.

\item \textbf{No se reciben trabajos fuera de fecha}, excepto con excusa según R.A.P.

\item \textbf{Asistencia obligatoria} con control por parte del profesor.

\item En exámenes escritos se esperan 15 minutos tras el inicio para ingresar; después se requiere supletorio según R.A.P.

\item \textbf{Casos de emergencia:} El profesor informará evacuaciones necesarias. No se responsabiliza por decisiones estudiantiles post-evacuación.
\end{itemize}